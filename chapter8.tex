\chapter{Gradient Fields and Path Integrals}

\section{Recitation Notes: Directional Derivatives}

\subsection{Definition}
The directional derivative of $f: \mathbb{R}^n \to \mathbb{R}$ at $x$ in the direction $v$ is:
\[ \frac{\partial f}{\partial v}(x) = \lim_{t \to 0} \frac{f(x + tv) - f(x)}{t} \]
If $f$ is differentiable at $x$, then:
\[ \frac{\partial f}{\partial v}(x) = f'(x)v = \langle \nabla f(x), v \rangle \]

\subsection{Critical Points}
\begin{theorem}
If $x^*$ is a local extremum (min or max) of a differentiable function $f$, then $\nabla f(x^*) = 0$.
Proof: For every direction $i$, the partial derivative $\frac{\partial f}{\partial x_i}$ must be 0.
\end{theorem}

\section{Lecture Notes: Path Integrals}

\subsection{Vector Fields}
\begin{definition}[Vector Field]
A function $g: \mathbb{R}^n \to \mathbb{R}^n$ is called a vector field.
Example: $g(x, y) = \begin{pmatrix} x^2 - y^2 - 4 \\ 2xy \end{pmatrix}$.
\end{definition}

\begin{definition}[Gradient Field]
A field $g$ is a \textbf{gradient field} (or conservative field) if there exists a scalar function $f: \mathbb{R}^n \to \mathbb{R}$ (called potential) such that $g = \nabla f = (f')^T$.
\end{definition}

\subsection{Path Integrals (Line Integrals)}
Let $g: \mathbb{R}^n \to \mathbb{R}^n$ be a vector field and $\gamma: [a, b] \to \mathbb{R}^n$ be a smooth path.
The integral of $g$ along $\gamma$ is defined as:
\[ \int_\gamma \langle g, d\gamma \rangle := \int_a^b \langle g(\gamma(t)), \gamma'(t) \rangle dt \]

\subsection{Fundamental Theorem of Line Integrals}
\begin{theorem}
If $g = \nabla f$ is a gradient field, then for any path $\gamma$ from $A$ to $B$:
\[ \int_\gamma \langle \nabla f, d\gamma \rangle = f(\gamma(b)) - f(\gamma(a)) = f(B) - f(A) \]
This means the integral depends only on the endpoints, not the path itself (Path Independence).
\end{theorem}

\subsection{Example}
Let $g = \nabla(x^2+y^2) = \begin{pmatrix} 2x \\ 2y \end{pmatrix}$.
Let $\gamma(t) = (\cos t, \sin t)$ for $t \in [0, 2\pi]$ (Unit circle).
Then $\langle g(\gamma(t)), \gamma'(t) \rangle = \langle (2\cos t, 2\sin t), (-\sin t, \cos t) \rangle = -2\cos t \sin t + 2\sin t \cos t = 0$.
Integral is 0.
Alternatively, using the potential $f(x, y) = x^2+y^2$:
End point = Start point, so $f(B) - f(A) = 0$.
