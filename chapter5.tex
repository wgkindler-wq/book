\chapter{Integration}

\section{Lecture Notes: The Riemann Integral}

\subsection{Defined on Rectangles}
Let $C = [a_1, b_1] \times \dots \times [a_n, b_n] \subseteq \mathbb{R}^n$ be a box (or rectangle).
The volume of $C$ is defined as $\text{vol}(C) = \prod_{i=1}^n (b_i - a_i)$.

\begin{definition}[Partition]
A partition $P$ of $C$ is a tuple $P = (P_1, \dots, P_n)$ where each $P_i$ is a partition of the interval $[a_i, b_i]$.
This induces a grid of sub-rectangles $J \in P$.
The norm of the partition is $|P| = \max_{j} |J_j|$ (maximum diameter of sub-rectangles).
\end{definition}

\subsection{Riemann Sums and Integrability}
For a bounded function $f: C \to \mathbb{R}$ and a partition $P$, we define the upper and lower Darboux sums:
\[ \overline{S}_P(f) = \sum_{J \in P} (\sup_{x \in J} f(x)) \cdot \text{vol}(J) \]
\[ \underline{S}_P(f) = \sum_{J \in P} (\inf_{x \in J} f(x)) \cdot \text{vol}(J) \]

\begin{definition}[Integrable Function]
A function $f$ is \textbf{integrable} on $C$ if for every $\epsilon > 0$ there exists a partition $P$ such that:
\[ \overline{S}_P(f) - \underline{S}_P(f) < \epsilon \]
In this case, the unique number separating the lower and upper sums is the integral, denoted $\int_C f$.
\end{definition}

\subsection{Properties of the Integral}
\begin{enumerate}
    \item \textbf{Linearity:} $\int_C (\alpha f + \beta g) = \alpha \int_C f + \beta \int_C g$.
    \item \textbf{Monotonicity:} If $f \le g$ on $C$, then $\int_C f \le \int_C g$.
    \item \textbf{Indicator Function:} For a set $K \subseteq C$, if $1_K$ is integrable, we define the volume of $K$ as $\text{vol}(K) = \int_C 1_K$.
    \item \textbf{Additivity of Domain:} If $C = K \cup L$ with negligible overlap, $\text{vol}(K \cup L) = \text{vol}(K) + \text{vol}(L)$.
\end{enumerate}

\section{Recitation Notes: Area Calculation}

\subsection{Example: Area of a Triangle}
Let $\Delta = \{ (x, y) \in \mathbb{R}^2 \mid 0 \le x \le 1, 0 \le y \le x \}$.
We want to compute $\text{area}(\Delta) = \int_{[0,1]^2} 1_\Delta$.
We construct a partition $P$ of $[0, 1] \times [0, 1]$ using a uniform grid of size $N \times N$.
The sub-squares that intersect the diagonal (boundary of $\Delta$) contribute to the difference between upper and lower sums.
The number of such squares is $O(N)$, and their total area is $O(1/N)$.
Since $\lim_{N \to \infty} O(1/N) = 0$, the function is integrable.
Calculation of the sum yields:
\[ \text{area}(\Delta) = \lim_{N \to \infty} \sum_{i=1}^N \frac{i}{N} \cdot \frac{1}{N^2} \approx \frac{1}{2} \]
Specifically, the calculation shows $\text{Area} = \frac{1}{2} + o(1)$.
