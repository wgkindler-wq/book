\section{Geometric Properties}
A property $\mathcal{P}$ of subsets of $\mathbb{R}^n$ is called a \textbf{geometric property} (with respect to translations) if it is invariant under translation.
That is, if a set $K$ has property $\mathcal{P}$, then for any shift $y$, the set $K+y$ also has property $\mathcal{P}$.
\textbf{Examples:}
\begin{itemize}
    \item ``Being a sphere'' is a geometric property.
    \item ``Containing the origin'' is \textbf{not} a geometric property (since shifting it might move the origin outside).
\end{itemize}
This formalism allows us to distinguish between intrinsic properties of a shape (like volume, shape type) and extrinsic properties (like position).
