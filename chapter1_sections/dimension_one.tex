\section{Dimension One: A Geometric Look at the Real Line}

Although we assume familiarity with the real numbers $\mathbb{R}$, it is beneficial to revisit them from a geometric perspective to set the stage for higher dimensions. The real line is a set $\mathbb{R}$ of elements, conventionally called numbers. However, in geometry, their role depends on the context.

When we visualize $\mathbb{R}$ as a line, each element corresponds to a specific location. In this context, we refer to elements of $\mathbb{R}$ as \textbf{points}.

\subsection{Translations and Vectors}

Elements of $\mathbb{R}$ have another geometric interpretation: they can represent a \textbf{translation} or shift. For example, the number $2$ can be thought of as the operation of shifting the origin (or any other point) by $2$ units to the right.

Formally, for any fixed number $v \in \mathbb{R}$, we can define a translation operator $T_v: \mathbb{R} \to \mathbb{R}$ by:
\[ T_v(x) = x + v \]
This operator takes a point $x$ and shifts it by $v$. When we think of elements of $\mathbb{R}$ as defining these shift operators, it is often helpful to visualize them as \textbf{arrows} rather than points.

An arrow can be drawn starting at the origin $0$ and ending at $v$ to represent the shift $v$. However, the same shift operation can be applied to any point. Thus, we can draw an arrow starting at any point $x$ and ending at $x+v$ to represent the same translation $v$.
Formally, we can think of an arrow as a pair of points $(x, y)$, where $x$ is the \emph{tail} (or anchor) and $y$ is the \emph{head}. This arrow represents the displacement from $x$ to $y$.

Two arrows $(x, y)$ and $(x', y')$ are considered \textbf{equivalent} if they represent the same displacement, which means:
\[ y - x = y' - x' \]
This defines an equivalence relation on the set of all arrows. We define a \textbf{vector} to be an equivalence class of arrows under this relation.
In this way, a vector captures the abstract notion of "magnitude and direction" (in 1D, direction is just sign) independent of a starting location. The number $v = y-x$ uniquely identifies this vector.

\subsection{Geometric Objects}

The concept of defining a geometric object as an equivalence class extends beyond vectors.
Consider a shape, like an interval $[a, b] \subset \mathbb{R}$. If we shift this interval by some amount $v$, we get a new set $[a+v, b+v]$. Intuitively, these two sets represent the "same" geometric shape, just in different locations.

We can define a \textbf{shape} as an equivalence class of subsets of $\mathbb{R}$ under translation. Two sets $A, B \subset \mathbb{R}$ determine the same shape if there exists a translation $v$ such that:
\[ B = \{ a + v \mid a \in A \} \]
From this perspective, a vector is simply a special case of a shape: it is the shape formed by a single point (an equivalence class of singletons).

We can generalize this even further. Consider a pair of shapes, or more generally, a tuple of subsets of $\mathbb{R}$. We can apply a translation to the entire tuple component-wise.
We define a \textbf{geometric object} as an equivalence class of tuples of subsets of $\mathbb{R}$ under translation.
For instance, a pair of points $(A, B)$ can be viewed as a geometric object. If we shift both $A$ and $B$ by the same $v$, we get an equivalent pair. This captures the relative position of the sets within the tuple.
Vectors are thus a special case of geometric objects where the tuple consists of two sets, each containing a single point. An arrow $(x, y)$ corresponds to the tuple of sets $(\{x\}, \{y\})$. The equivalence class of this tuple under translation is precisely the vector $y-x$. This broad definition allows us to study properties that depend on the configuration of multiple shapes relative to each other, invariant under the absolute position in space.

A \textbf{geometric property} can be viewed in two equivalent ways. First, it is a property of tuples of sets which is invariant under translations. For example, the property "the distance between the two points is 5" is a geometric property of a pair of points, because shifting the pair does not change the distance.
Equivalently, it is a property of geometric objects themselves. Since all tuples in an equivalence class share the same invariant properties, we can ascribe the property to the class as a whole. Thus, "length" is a property of the vector (or interval) itself, independent of where it is located in space.

