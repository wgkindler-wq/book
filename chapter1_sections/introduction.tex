\section{Introduction: From High School Geometry to University Mathematics}

In high school geometry, we often distinguish between the intrinsic properties of a geometric object and its specific location in space. We intuitively understand that a triangle remains the "same" triangle, or a circle the "same" circle, regardless of where it is drawn. A familiar example of this principle is the treatment of vectors as ``arrows'': two arrows are considered equivalent if they possess the same length and direction, even if they are anchored at different points. This suggests that geometry is less about fixed locations and more about the relationships and transformations between them.

In this chapter, we formalize this intuition using the language of $\mathbb{R}^n$. We will see that vectors can be thought of as ``translations'' or ``shifts'' of the space, and that many geometric properties we are familiar with (like congruence) are simply properties that remain unchanged when we shift figures around.
