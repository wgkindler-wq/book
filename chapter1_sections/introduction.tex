\section{Introduction: From High School Geometry to University Mathematics}

When studying geometry in high school we distinguish between the intrinsic properties of a geometric object, and its specific location in space. We intuitively understand that a triangle remains the ``same'' triangle, and a circle the ``same'' circle, regardless of where they are located. On the other hand, when we ask questions about the relation within a collection of objects, for example whether a circle is contained in a triangle, then their relative position matters, and moving just one of them might change the answer. A familiar example of this principle is in the treatment of vectors: a vector can be represented as an ordered pair of points, an anchor point and a head, which together form an arrow shape starting at the anchor point and ending at the head. Two arrows are considered equivalent if they possess the same length and direction, even if they are anchored at different points. This discussion suggests that geometry is not about points or sets of points in space, but rather about the relationships between them, which are invariant under translations and perhaps also other operations.

In this chapter, we formalize this intuition and genralize it from the familiar setting of $\mathbb{R}^2$ and $\mathbb{R}^3$ to the general setting of $\mathbb{R}^n$. Indeed throughout this book while we will often use the familiar settings of $\mathbb{R}^2$ and $\mathbb{R}^3$ to illustrate concepts, we will always define them in the general setting of $\mathbb{R}^n$. However, to really start from the basics, we will begin with the simplest case of all, the one-dimensional space $\mathbb{R}^1$.
