\section[From High School Geometry to University Mathematics]{From High School Geometry to University Mathematics}

In this chapter we study some very basic concepts in high-dimensional geometry. Why do we consider geometry at all, if our main goal is to study calculus? In the calculus of a single variable, geometric intuition plays a crucial role. We understand the derivative as the slope of a tangent line, and the integral as the area under a curve. These geometric interpretations provide a mental map that guides us through formal proofs and calculations.

As we move to the calculus of several variables, we wish to retain this powerful advantage. We want to be able to use our geometric intuition to navigate the analysis of functions in higher dimensions. However, our innate intuition is limited to three dimensions. To work effectively in $\mathbb{R}^n$, we must therefore extend our geometric instincts. We do this by taking familiar concepts from the plane and space—such as points, vectors, lines, and distances—and generalizing them to $n$ dimensions.

This chapter is dedicated to establishing this geometric foundation. Our primary aim is to harness our low-dimensional intuition to make high-dimensional calculus natural and accessible. However, as we progress, we will discover that the relationship between these fields is two-way. Just as geometry provides intuition for calculus, the powerful tools of calculus will ultimately allow us to explore and understand the geometry of high-dimensional spaces with a depth that intuition alone cannot provide.

When studying geometry in high school we distinguish between the intrinsic properties of a geometric object, and its specific location in space. We intuitively understand that a triangle remains the ``same'' triangle, and a circle the ``same'' circle, regardless of where they are located. On the other hand, when we ask questions about the relation within a collection of objects, for example whether a circle is contained in a triangle, then their relative position matters, and moving just one of them might change the answer. A familiar example of this principle is in the treatment of vectors: a vector can be represented as an ordered pair of points, an anchor point and a head, which together form an arrow shape starting at the anchor point and ending at the head. Two arrows are considered equivalent if they possess the same length and direction, even if they are anchored at different points. This discussion suggests that geometry is not about points or sets of points in space, but rather about the relationships between them, which are invariant under translations and perhaps also other operations.

In this chapter, we formalize this intuition and genralize it from the familiar setting of $\mathbb{R}^2$ and $\mathbb{R}^3$ to the general setting of $\mathbb{R}^n$. Indeed throughout this book we often use the familiar settings of $\mathbb{R}^2$ and $\mathbb{R}^3$ to illustrate concepts, while defining them in the general setting of $\mathbb{R}^n$. However to really start from the basics, we will begin with the simplest case of all, the one-dimensional space $\mathbb{R}^1$.
